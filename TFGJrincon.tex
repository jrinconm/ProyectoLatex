%%%%%%%%%%%%%%%%%%%%%%%%%%%%%%%%%%%%%%%%%%%%%%%%%%%%%%%%%%%%%%%%%%%%%%%%%%%%%%%
%                       CARREGA DE LA CLASSE DE DOCUMENT                      %
%                                                                             %
% Les opcions admissibles son:                                                %
%      12pt / 11pt            (cos dels tipus de lletra; no feu servir 10pt)  %
%                                                                             %
% catalan/spanish/english     (llengua principal del treball)                 %
%                                                                             % 
% french/italian/german...    (si necessiteu fer servir alguna altra llengua) %
%                                                                             %
% listoffigures               (El document inclou un Index de figures)        %
% listoftables                (El document inclou un Index de taules)         %
% listofquadres               (El document inclou un Index de quadres)        %
% listofalgorithms            (El document inclou un Index d'algorismes)      %
%                                                                             %
%%%%%%%%%%%%%%%%%%%%%%%%%%%%%%%%%%%%%%%%%%%%%%%%%%%%%%%%%%%%%%%%%%%%%%%%%%%%%%%

\documentclass[11pt,spanish,listoffigures,listoftables]{tfgetsinf}

%%%%%%%%%%%%%%%%%%%%%%%%%%%%%%%%%%%%%%%%%%%%%%%%%%%%%%%%%%%%%%%%%%%%%%%%%%%%%%%
%                     CODIFICACIO DEL FITXER FONT                             %
%                                                                             %
%    windows fa servir normalment 'ansinew'                                   %
%    amb linux es possible que siga 'latin1' o 'latin9'                       %
%    Pero el mes recomanable es fer servir utf8 (unicode 8)                   %
%                                          (si el vostre editor ho permet)    % 
%%%%%%%%%%%%%%%%%%%%%%%%%%%%%%%%%%%%%%%%%%%%%%%%%%%%%%%%%%%%%%%%%%%%%%%%%%%%%%%

\usepackage[utf8]{inputenc} 

%%%%%%%%%%%%%%%%%%%%%%%%%%%%%%%%%%%%%%%%%%%%%%%%%%%%%%%%%%%%%%%%%%%%%%%%%%%%%%%
%                        ALTRES PAQUETS I DEFINICIONS                         %
%                                                                             %
% Carregueu aci els paquets que necessiteu i declareu les comandes i entorns  %
%                                          (aquesta seccio pot ser buida)     %
%%%%%%%%%%%%%%%%%%%%%%%%%%%%%%%%%%%%%%%%%%%%%%%%%%%%%%%%%%%%%%%%%%%%%%%%%%%%%%%



%%%%%%%%%%%%%%%%%%%%%%%%%%%%%%%%%%%%%%%%%%%%%%%%%%%%%%%%%%%%%%%%%%%%%%%%%%%%%%%
%                        DADES DEL TREBALL                                    %
%                                                                             %
% titol, alumne, tutor i curs academic                                        %
%%%%%%%%%%%%%%%%%%%%%%%%%%%%%%%%%%%%%%%%%%%%%%%%%%%%%%%%%%%%%%%%%%%%%%%%%%%%%%%

\title{Gestor de turnos  \\
         de teletrabajo}
\author{Julián Rincón Morales}
\tutor{Pau Miñana Climent}
\curs{2020-2021}

%%%%%%%%%%%%%%%%%%%%%%%%%%%%%%%%%%%%%%%%%%%%%%%%%%%%%%%%%%%%%%%%%%%%%%%%%%%%%%%
%                     PARAULES CLAU/PALABRAS CLAVE/KEY WORDS                  %
%                                                                             %
% Independentment de la llengua del treball, s'hi han d'incloure              %
% les paraules clau i el resum en els tres idiomes                            %
%%%%%%%%%%%%%%%%%%%%%%%%%%%%%%%%%%%%%%%%%%%%%%%%%%%%%%%%%%%%%%%%%%%%%%%%%%%%%%%

\keywords{teletreball, organització, torn, node, quasar, sequelize, mariadb, docker, express, rest api} % Paraules clau 
         {teletrabajo, organizacion, turno, node, quasar, sequelize, mariadb, docker, express, rest api}              % Palabras clave
         {telework,  setup, turn, node, quasar, sequelize, mariadb, docker, express, rest api}        % Key words

%%%%%%%%%%%%%%%%%%%%%%%%%%%%%%%%%%%%%%%%%%%%%%%%%%%%%%%%%%%%%%%%%%%%%%%%%%%%%%%
%                              INICI DEL DOCUMENT                             %
%%%%%%%%%%%%%%%%%%%%%%%%%%%%%%%%%%%%%%%%%%%%%%%%%%%%%%%%%%%%%%%%%%%%%%%%%%%%%%%

\begin{document}

%%%%%%%%%%%%%%%%%%%%%%%%%%%%%%%%%%%%%%%%%%%%%%%%%%%%%%%%%%%%%%%%%%%%%%%%%%%%%%%
%              RESUMS DEL TFG EN VALENCIA, CASTELLA I ANGLES                  %
%%%%%%%%%%%%%%%%%%%%%%%%%%%%%%%%%%%%%%%%%%%%%%%%%%%%%%%%%%%%%%%%%%%%%%%%%%%%%%%

\begin{abstract}
????
\end{abstract}
\begin{abstract}[spanish]
Gestor de turnos de teletrabajo es una herramienta para organizar los turnos en los que el personal de los distintos departamentos de una empresa se turnan para teletrabajar.
Para el cliente básico, se ofrece un calendario donde puede con firmar los turnos que le son asignados o solicitar cambio a otro compañero.
También tiene visibilidad sobre los turnos de su propia area para poder solicitar el cambio.
También puede obtener una copia del calendario en pdf para imprimirla posteriormente o almacenarla.
El gestor del departamento puede obtener un informe con graficos del personal que está a su cargo.
El administrador de la aplicación tiene un interfaz para poder modificar los valores de la aplicación.
\end{abstract}
\begin{abstract}[english]
????
\end{abstract}

%%%%%%%%%%%%%%%%%%%%%%%%%%%%%%%%%%%%%%%%%%%%%%%%%%%%%%%%%%%%%%%%%%%%%%%%%%%%%%%
%                              CONTINGUT DEL TREBALL                          %
%%%%%%%%%%%%%%%%%%%%%%%%%%%%%%%%%%%%%%%%%%%%%%%%%%%%%%%%%%%%%%%%%%%%%%%%%%%%%%%

\mainmatter

%%%%%%%%%%%%%%%%%%%%%%%%%%%%%%%%%%%%%%%%%%%%%%%%%%%%%%%%%%%%%%%%%%%%%%%%%%%%%%%
%                                  INTRODUCCIO                                %
%%%%%%%%%%%%%%%%%%%%%%%%%%%%%%%%%%%%%%%%%%%%%%%%%%%%%%%%%%%%%%%%%%%%%%%%%%%%%%%

\chapter{Introducci\'o}

Al incorporar el teletrabajo en nuestra forma de vida, muchas empresas que no tenían ningún sistema de turnos porque no les hacía falta, han visto que no disponen de herramientas para coordinar los diversos grupos de trabajo.
Muchas optan por usar el calendario del correo o  una hoja excel, en ocasiones generando descoordinación al no tener un sistema fácil donde poder marcar y modificar los días en los que el personal debe acudir de forma presencial.
También es posible que no se distribuyan de forma equitativa los turnos, pudiendo generar malestar entre los compañeros.

% \section{Motivaci\'o}
%
%Lograr una aplicación que muestre  
%
\section{Objectius}

El objetivo de este proyecto es crear una aplicación que de forma rápida permita configurar las jornadas de teletrabajo, y dotar a los responsables de los mismos estadísticas del personal a su cargo.

\section{Estructura de la mem\`oria}

En el capítulo 1 presento las principales herramientas y mejorar logradas en el software.
En el capítulo 2 debería hacer el estudio de mercado, dafo, recursos, etc.
En el capítulo 3 los requisitos.
En el capítulo 4 los diagramas.
En el capítulo 5 la codificación, el por qué , etc.
En el capítulo 6 el despligue.
En el capítulo 7 las herramientas de apoyo.
En el capítulo 8 las conclusiones


%\section{Notes bibliografiques} %%%%% Opcional

%????? ????????????? ????????????? ????????????? ????????????? ?????????????

%%%%%%%%%%%%%%%%%%%%%%%%%%%%%%%%%%%%%%%%%%%%%%%%%%%%%%%%%%%%%%%%%%%%%%%%%%%%%%%
%                         CAPITOLS (tants com calga)                          %
%%%%%%%%%%%%%%%%%%%%%%%%%%%%%%%%%%%%%%%%%%%%%%%%%%%%%%%%%%%%%%%%%%%%%%%%%%%%%%%

\chapter{Estado del arte}

El ORM (object-relational mapping) Sequelize, desde su versión 4, permite el uso de clases ES6, por lo que se ha usado una clase básica con los parametros comunes, eliminando codigo duplicado.
Se ha integrado CI/CD en la creación de esta documentación, a través de github actions, compilando el documento al realizar un push en el respositorio.

\section{?? ???? ???? ? ?? ??}

????? ????????????? ????????????? ????????????? ????????????? ?????????????

\chapter{Estudio de viabilidad. Método DAFO}

????? ????????????? ????????????? ????????????? ????????????? ?????????????

\section{Estudio de mercado}

????? ????????????? ????????????? ????????????? ????????????? ?????????????

\section{Planificación temporal}

????? ????????????? ????????????? ????????????? ????????????? ?????????????

\chapter{Análisis de requisitos}

????? ????????????? ????????????? ????????????? ????????????? ????????????? 

\section{?? ???? ???? ? ?? ??}

????? ????????????? ????????????? ????????????? ????????????? ?????????????

\chapter{Diseño}

????? ????????????? ????????????? ????????????? ????????????? ????????????? 

\section{Diseño Conceptual Entidad Relación}

????? ????????????? ????????????? ????????????? ????????????? ?????????????

\section{Diseño Lógico Relacional o Paso a tablas}

????? ????????????? ????????????? ????????????? ????????????? ?????????????

\section{Diseño Físico o Diagrama Mysql}

????? ????????????? ????????????? ????????????? ????????????? ?????????????

\section{Orientación a objetos}

????? ????????????? ????????????? ????????????? ????????????? ?????????????

\section{Mapa Web}

????? ????????????? ????????????? ????????????? ????????????? ?????????????

\section{Mockups}

????? ????????????? ????????????? ????????????? ????????????? ?????????????

\chapter{Codificación}

????? ????????????? ????????????? ????????????? ????????????? ?????????????

\section{?? ???? ???? ? ?? ??}

????? ????????????? ????????????? ????????????? ????????????? ?????????????

\chapter{Despliegue}

????? ????????????? ????????????? ????????????? ????????????? ?????????????

\section{?? ???? ???? ? ?? ??}

????? ????????????? ????????????? ????????????? ????????????? ?????????????

\chapter{Herramientas de apoyo}

????? ????????????? ????????????? ????????????? ????????????? ?????????????

\section{?? ???? ???? ? ?? ??}

????? ????????????? ????????????? ????????????? ????????????? ?????????????


%%%%%%%%%%%%%%%%%%%%%%%%%%%%%%%%%%%%%%%%%%%%%%%%%%%%%%%%%%%%%%%%%%%%%%%%%%%%%%%
%                                 CONCLUSIONS                                 %
%%%%%%%%%%%%%%%%%%%%%%%%%%%%%%%%%%%%%%%%%%%%%%%%%%%%%%%%%%%%%%%%%%%%%%%%%%%%%%%

\chapter{Conclusions}

????? ????????????? ????????????? ????????????? ????????????? ????????????? 

%%%%%%%%%%%%%%%%%%%%%%%%%%%%%%%%%%%%%%%%%%%%%%%%%%%%%%%%%%%%%%%%%%%%%%%%%%%%%%%
%                                BIBLIOGRAFIA                                 %
%%%%%%%%%%%%%%%%%%%%%%%%%%%%%%%%%%%%%%%%%%%%%%%%%%%%%%%%%%%%%%%%%%%%%%%%%%%%%%%

\begin{thebibliography}{10}

\bibitem{WAR}
   Issue del github de sequelize donde está la definición y ejemplos de clases para definir los modelos.
   \newblock Consultat a 
   \url{https://github.com/sequelize/sequelize/issues/6524}.

\bibitem{WAR}
   Howto creación de api con Node, Sequelize, Express y mysql.
   \newblock Consultat a 
   \url{https://bezkoder.com/node-js-express-sequelize-mysql}.

\bibitem{WAR}
   Documentación oficial sequelize.
   \newblock Consultat a 
   \url{https://sequelize.org/master}.

\bibitem{WAR}
   Documentación oficial sequelize.
   \newblock Consultat a 
   \url{https://sequelize.org/master}.

\bibitem{WAR}
   Documentación oficial componente QCalendar.
   \newblock Consultat a 
   \url{https://quasarframework.github.io/quasar-ui-qcalendar/docs}.

\bibitem{WAR}
   Documentación oficial Express.
   \newblock Consultat a 
   \url{https://expressjs.com/es/guide/routing.html}.

%%%%%%%%%%%%%%%%%%%%%%%%%%%%%%%%%%%%%%%%%%%%%%%%%%%%%%%%%%%%%%%%%%%%%%%%%%%%%%%
% MODEL D'ARTICLE                                                             %
%%%%%%%%%%%%%%%%%%%%%%%%%%%%%%%%%%%%%%%%%%%%%%%%%%%%%%%%%%%%%%%%%%%%%%%%%%%%%%%
\bibitem{light}
   Jennifer~S. Light.
   \newblock When computers were women.
   \newblock \textit{Technology and Culture}, 40:3:455--483, juliol, 1999.

%%%%%%%%%%%%%%%%%%%%%%%%%%%%%%%%%%%%%%%%%%%%%%%%%%%%%%%%%%%%%%%%%%%%%%%%%%%%%%%
% MODEL DE LLIBRE                                                             %
%%%%%%%%%%%%%%%%%%%%%%%%%%%%%%%%%%%%%%%%%%%%%%%%%%%%%%%%%%%%%%%%%%%%%%%%%%%%%%%
\bibitem{ifrah}
   Georges Ifrah.
   \newblock \textit{Historia universal de las cifras}.
   \newblock Espasa Calpe, S.A., Madrid, sisena edició, 2008.

%%%%%%%%%%%%%%%%%%%%%%%%%%%%%%%%%%%%%%%%%%%%%%%%%%%%%%%%%%%%%%%%%%%%%%%%%%%%%%%
% MODEL D'URL                                                                 %
%%%%%%%%%%%%%%%%%%%%%%%%%%%%%%%%%%%%%%%%%%%%%%%%%%%%%%%%%%%%%%%%%%%%%%%%%%%%%%%
\bibitem{WAR}
   Comunicat de premsa del Departament de la Guerra, 
   emés el 16 de febrer de 1946. 
   \newblock Consultat a 
   \url{http://americanhistory.si.edu/comphist/pr1.pdf}.

\end{thebibliography}
\cleardoublepage

%%%%%%%%%%%%%%%%%%%%%%%%%%%%%%%%%%%%%%%%%%%%%%%%%%%%%%%%%%%%%%%%%%%%%%%%%%%%%%%
%                           APÈNDIXS  (Si n'hi ha!)                           %
%%%%%%%%%%%%%%%%%%%%%%%%%%%%%%%%%%%%%%%%%%%%%%%%%%%%%%%%%%%%%%%%%%%%%%%%%%%%%%%

\APPENDIX

%%%%%%%%%%%%%%%%%%%%%%%%%%%%%%%%%%%%%%%%%%%%%%%%%%%%%%%%%%%%%%%%%%%%%%%%%%%%%%%
%                         LA CONFIGURACIO DEL SISTEMA                         %
%%%%%%%%%%%%%%%%%%%%%%%%%%%%%%%%%%%%%%%%%%%%%%%%%%%%%%%%%%%%%%%%%%%%%%%%%%%%%%%

\chapter{Configuració del sistema}

????? ????????????? ????????????? ????????????? ????????????? ?????????????

\section{Fase d'inicialització}

????? ????????????? ????????????? ????????????? ????????????? ?????????????

\section{Identificació de dispositius}

????? ????????????? ????????????? ????????????? ????????????? ?????????????

%%%%%%%%%%%%%%%%%%%%%%%%%%%%%%%%%%%%%%%%%%%%%%%%%%%%%%%%%%%%%%%%%%%%%%%%%%%%%%%
%                               ALTRES  APÈNDIXS                              %
%%%%%%%%%%%%%%%%%%%%%%%%%%%%%%%%%%%%%%%%%%%%%%%%%%%%%%%%%%%%%%%%%%%%%%%%%%%%%%%


\chapter{??? ???????????? ????}

????? ????????????? ????????????? ????????????? ????????????? ????????????? 



%%%%%%%%%%%%%%%%%%%%%%%%%%%%%%%%%%%%%%%%%%%%%%%%%%%%%%%%%%%%%%%%%%%%%%%%%%%%%%%
%                              FI DEL DOCUMENT                                %
%%%%%%%%%%%%%%%%%%%%%%%%%%%%%%%%%%%%%%%%%%%%%%%%%%%%%%%%%%%%%%%%%%%%%%%%%%%%%%%

\end{document}
